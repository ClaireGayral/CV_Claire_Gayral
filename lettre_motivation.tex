\documentclass[a4paper,12pt]{article}

\usepackage[T1]{fontenc}
\usepackage[utf8]{inputenc}
\usepackage[intlimits]{amsmath}
\usepackage{amsfonts,amssymb,amscd,amstext}


%%%% tikz library : %%%%
\usepackage{tikz}
\usetikzlibrary{tikzmark,calc,arrows,shapes,decorations.pathreplacing}
\usetikzlibrary{patterns,plotmarks}
\usetikzlibrary{decorations.pathreplacing,decorations.markings, arrows}
\newcommand{\midarrow}{\tikz \draw[-triangle 90] (0,0) -- +(.1,0);}
\usepackage{pgfplots}
\usepackage{caption,subcaption}
\tikzset{every picture/.style={remember picture}}


\usepackage{multicol}
\usepackage{enumerate}
\usepackage{array}
\usepackage{theorem}
\newtheorem{remark}{Remark}
\newtheorem{theorem}{Theorem}[subsection]

\newcommand{\aaa}{\alpha}
\newcommand{\bb}{\beta}


%ajouts ulysse
\newcommand{\Int}{\int\limits}
\newcommand{\om}{$\!\!^\sharp\,\,$}
\newcommand{\di}{$\!\!^\star\,\,$}
%\pagestyle{empty}
\newcommand{\ligneH}{\noindent\makebox[\linewidth]{\rule{\textwidth}{.5pt}}} %\paperwidth

%%%% mise en page : 
\pagestyle{plain} \setlength{\textwidth}{19cm}
\setlength{\textheight}{27cm} \setlength{\topmargin}{-3cm}
\setlength{\oddsidemargin}{-1,2cm}
\setlength{\evensidemargin}{+1,2cm}

\title{Lettre de motivation}
\author{Claire Gayral}
\date{}

\begin{document}

\maketitle

Cher Monsieur Seban, cher Monsieur Redko,

Votre offre de thèse pluridisciplinaire, sur le développement de méthodes d'apprentissage machine pour comprendre le vieillissement de la m\oe{}lle osseuse lié à une exposition aux rayons cosmiques, semble tout à fait correspondre à mon projet professionnel. Je me permets donc de vous exposer les raisons qui me poussent à postuler à cette offre, qui est une opportunité de choix pour moi, parce qu'elle fait écho à différentes thématiques qui me plaisent. \\

La première chose qu'il me semble nécessaire à motiver est mon envie de faire une thèse. L'historique débute lors d'un stage facultatif, fait dans l'entreprise LumenAI, en fin de première année de Master, pendant lequel j'ai été encadrée par Sébastien Lousteau et Camille Saumard. Ce stage m'a permis de réaliser que la fouille de données avec les compétences que j'avais allait très rapidement m'ennuyer, car la réalité de l'entreprise n'en fait pas un milieu propice à l'auto-formation. 

Il m'était donc nécessaire d'apprendre à apprendre par moi-même. D'autant plus que ma curiosité pour les mathématiques se cachant derrière les modèles d'apprentissage restait à assouvir.
C'est ainsi qu'est né ce projet de thèse, par son aspect de formation à la recherche. 
Ce projet sur 3 ans est aussi un défi de par les enjeux de publication, et de façon plus globale par le modeste apport à la science qu'il apportera. Je ne suis pas encore complètement figée dans mon projet professionnel après cette étape : chercher un poste d'enseignant-chercheur, ou partir dans un service Recherches et Développement d'une entreprise. Je pense avoir le temps de maturer cela au cours de la thèse (et voir les opportunités à la fin de celle-ci !). 

Pour ma deuxième année de Master, j'ai donc choisi quitter la filière professionnalisante, au profit de la filière recherche - année validée en étant 7$^{\text{ième}}$ de ma promotion. 
J'ai enchaîné sur 4 mois de stage de recherche co-encadré par J.Chiquet et F.Picard, durant lesquels je me suis familiarisée avec les notions biologiques de génomique en cellule unique, et j'ai nettoyé et analysé un jeu de données, dans l'objectif de comprendre des mécanismes de régulation qui expliqueraient la différentiation de lymphocyte. Malheureusement, ces données avaient trop peu de signal pour les utiliser dans un cadre d'intégration (3 mesures par groupe à identifier pour l'expression génomique, et 27 mesures pour 23 groupes dans la régulation). Ce stage a débouché, comme il en avait été convenu, sur un contrat doctoral, portant sur le sujet "intégration de données de séquençage en cellule unique". 
Par intégration, j'entends trouver des plans de projection dans lesquels les données aurait une distance faible, comme l'Analyse Canonique des Corrélation (CCA) cherche un plan de projection où la distance "corrélation" est faible. J'ai commencé m'intéresser à cette méthode classique d'intégration, et ses extensions à noyaux, et probabilistes, puis j'ai regardé quelles métriques pourraient être utilisées pour intégrer ces données avec des distributions si particulières. Entre autre, et je pense que vous comprendrez pourquoi j'insiste sur ce point, j'ai consacré quelques mois à me familiariser d'un point de vue théorique avec les distances, de Wasserstein, dans leur forme semi-discrète (l'objectif étant de remplacer la perte de maximum de vraissemblance dans la CCA par une perte de minimum de Wasserstein). En parallèle de ces travaux bibliographiques, j'ai donné des enseignements, sur un peu plus de 120 heures équivalent TD depuis le mois de janvier. 
Je viens donc de finir cette année de thèse, qui a été très difficile, car nous n'avons pas réussi à trouver une communication constructive avec un de mes directeur, ce pourquoi il est préférable de changer d'orientation. 
Néanmoins, ni mes compétences, ni les raisons qui me poussaient à entreprendre un doctorat n'ont été remises en cause (je vous laisse le soin de contacter mon co-directeur avec qui la communication est limpide, J. Chiquet, ou encore mon encadrant de stage de M1, S. Lousteau). 
Les connaissances apprises et les habitudes de travail acquises durant cette année sont des atouts qui me rendent très compétitive pour un autre projet de thèse, et font de cette expérience non un échec, mais bien un tremplin vers un projet qui me conviendrait mieux. 
%
Ma formation m'aurait d'ailleurs permis de postuler sans cela : les bases de traitement du signal acquis en M1, ainsi que le cours de \textit{Machine Learning} (du master \textit{Data science} de Lyon 1), suivis en autodidacte en M2 m'ont donné le bagage nécessaire pour me lancer dans ce projet. Enfin, mon expérience dans l'entreprise LumenAI fait preuve de mon niveau de programmation, et de ma capacité à prendre en main les outils de Machine Learning en python. En plus de ces compétences, j'ai appris cette année à utiliser un serveur distant, dans le cadre d'un projet d'intégration de données omiques (qui n'étaient pas en cellule unique), pour lancer mes scripts R. \\
%

Pourquoi cette offre en particulier ? Le fait que ce soit un travail pluridisciplinaire me plaît : l'occasion de découvrir des curiosités biologiques est un point qui me parle, c'est d'ailleurs pour cela que j'avais commencé mes études par une année de biologie, et c'est la même raison qui m'a poussée à choisir la spécialité "application à la biologie et à la médecine" du Master Maths en Action. Cette application des mathématiques à des problématiques médicales me paraît constructive pour la science (bien plus que de travailler à améliorer des algorithmes de publicité ciblée !). 
D'un point de vue technique, je peux vous affirmer que je serai compétente pour répondre aux attentes pour ce projet. Dans le descriptif de l'offre, vous évoquiez deux axes distincts pour ce projet de thèse. Le deuxième axe, sur des méthodes de transferts, fait écho à ce que j'ai pu lire sur l'intégration de données omiques, qui était souvent dans le but d'utiliser un a priori (expression de gènes), pour conclure sur un autre aspect (une régulation liée à l'accessibilité de l'ADN par exemple). Le premier axe, avec lequel je suis moins familier, porterait sur des méthodes de super-résolution. Cela fait écho aux rudiments de traitement de l'image que j'ai eu en première année de Master, et j'ai hâte de découvrir ce domaine ! %, et le fait de ne pas avoir creusé cet intérêt était purement par manque d'opportunité. 
%
Par ailleurs, le fait que les données soit acquises et disponibles dès le début du travail est, à mes yeux, une grande qualité : cela permet de cibler la ou les problématiques biologiques et pouvoir tester les méthodes adaptées sur des données concrètes, avec toutes leurs particularités. Enfin, je suis originaire de la région, et nous souhaitons nous installer de nouveau ici avec mon conjoint. C'est un aspect positif, non décisif, mais qui mérité d'être mentionné. 
\\
%

C'est ainsi que je me présente pour mener à bien ce projet avec vous. Je suis certaine qu'il sera facile de construire une relation de confiance dans le travail, dès lors que la communication est claire, et les directions scientifiques prises dans un esprit de collaboration. Je reste donc à votre entière disposition pour construire cette thèse, en espérant que ma candidature puisse correspondre à vos attentes. 



\hspace{3cm}

\textbf{Contenu mail} : Monsieur Seban, Monsieur Redko, 

Je vous contacte au sujet de l'offre de thèse intitulée "Machine learning for understanding the aging of a human bone due to space exposure", qui pourrait tout à fait correspondre à mon projet professionnel. Comme exigé dans l'offre, je vous joins donc ma lettre de motivation et mon CV, en anglais et français, ainsi qu'une \textbf{lettre de recommandation de mon responsable de master} (si tu veux bien :) ).%, et une de la part d'un de mes directeurs actuel.% ainsi que de mon encadrant de stage de M1 (fouille de données en python).  
Je vous laisse les contacts de différentes personnes, qui ont eu l'occasion de m'encadrer, qui ont gentiment proposé d'être référents : 
- Julien Chiquet : 
- Sébastien Lousteau :
- Clément Marteau : 
- Camille Saumard : 




\end{document}
