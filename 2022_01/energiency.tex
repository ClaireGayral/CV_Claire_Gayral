\documentclass[a4paper,11pt]{article}
\renewcommand{\baselinestretch}{0.94} 
\usepackage[T1]{fontenc}
\usepackage[utf8]{inputenc}
\usepackage[intlimits]{amsmath}
\usepackage{amsfonts,amssymb,amscd,amstext}


%%%% tikz library : %%%%
\usepackage{tikz}
\usetikzlibrary{tikzmark,calc,arrows,shapes,decorations.pathreplacing}
\usetikzlibrary{patterns,plotmarks}
\usetikzlibrary{decorations.pathreplacing,decorations.markings, arrows}
\newcommand{\midarrow}{\tikz \draw[-triangle 90] (0,0) -- +(.1,0);}
\usepackage{pgfplots}
\usepackage{caption,subcaption}
\tikzset{every picture/.style={remember picture}}


\usepackage{multicol}
\usepackage{enumerate}
\usepackage{array}
\usepackage{theorem}
\newtheorem{remark}{Remark}
\newtheorem{theorem}{Theorem}[subsection]

\newcommand{\aaa}{\alpha}
\newcommand{\bb}{\beta}


%ajouts ulysse
\newcommand{\Int}{\int\limits}
\newcommand{\om}{$\!\!^\sharp\,\,$}
\newcommand{\di}{$\!\!^\star\,\,$}
%\pagestyle{empty}
\newcommand{\ligneH}{\noindent\makebox[\linewidth]{\rule{\textwidth}{.5pt}}} %\paperwidth

%%%% mise en page : 
\pagestyle{plain} \setlength{\textwidth}{17,6cm}
\setlength{\textheight}{25cm} \setlength{\topmargin}{-2cm}
\setlength{\oddsidemargin}{-0.9cm}
\setlength{\evensidemargin}{+0.9cm}

\title{Lettre de motivation}
\author{Claire Gayral}
\date{}

\begin{document}


Bonjour Madame Le Jort,\\

Je me permets de prendre contact avec vous au sujet de votre recherche de Data Scientist à Rennes. En effet, je viens de finaliser une formation complémentaire de programmation python chez OpenClassRooms, et un tel poste correspond tout à fait au métier auquel je me destinais. Voilà pourquoi. 

% présentation stage M1
J'ai découvert quelques applications du Machine Learning lors de mon stage facultatif de fin de première année de master, encadré par S. Lousteau et C. Brunet-Saumard.
En traitant des problématiques de détections de fraudes téléphoniques, et de maintenance prédictive pour l'entreprise LumenAI, j'ai alors découvert de nombreux outils, tant mathématiques (graphes, transport optimal, clustering, ...) que de programmation python (pandas, NetworkX, ...), et cela en développant deux "Proof Of Concepts", sur les trois mois du stage.

% Stage M2 et thèse 1
Pour ma deuxième année de Master, j'ai choisi de quitter la filière professionnalisante, au profit de la filière recherche, avec un plus grand approfondissement mathématiques. Le master a été validé en étant 7$^{\text{ième}}$ de promotion. En parallèle, j'ai suivi les cours de Machine Learning et Probabilistic Graphical Models d'un autre master, en autodidacte. 
L'année s'est conclue avec un stage co-encadré par J. Chiquet et F. Picard, portant sur l'analyse de mesures d'expression et régulation génomique de lymphocytes, issus d'un séquençage en cellule unique. Ce travail technique, fait en $R$ n'a pu aboutir à la publication prévue, sur des mécanismes de régulation responsables de la différentiation des lymphocytes, par manque de signal dans les données. Néanmoins, ce travail a été une parfaite introduction aux notions biologiques de génomique en cellule unique. 

J'ai continué ce travail bibliographique sur l'intégration de données de séquençage en cellule unique l'année suivante, dans le même laboratoire.
%
Par intégration, j'entends trouver des plans de projection dans lesquels la distance est faible, comme l'Analyse Canonique des Corrélation (CCA) projette les données dans un espace où la distance "corrélation" est petite. Partant de cette méthode classique, j'ai étudié ses différentes extensions (à noyaux, probabiliste) puis j'ai cherché à remplacer la métrique par une mesure qui prendrait mieux en compte la distribution si particulière des données (différentes divergences, transport optimal), liées au séquençage en cellule unique. 
En parallèle de cette veille scientifique, j'ai mené un projet annexe plus concret, pour lequel il a fallu apprendre, en toute autonomie, à utiliser le serveur de calcul du laboratoire, avec un suivi git et toutes les bonnes pratiques nécessaires à un projet commun avec d'autres personnes.
% cours :
En plus de ces travaux, j'ai dispensé une centaine d'heures de TD pour l'UFR de mathématiques de l'UCBL. 

%% Openclassrooms
A la fin de ce contrat, j'ai décidé de perfectionner ma production de code, par une formation "ingénieur Machine Learning" proposée par OpenClassRooms. Cette formation en ligne est basée sur des projets avec différentes problématiques : prédiction d'un score pour les aliments, prédiction de la consommation électrique de bâtiments, segmentation de clients, catégorisation de questions et prédiction de tags associés, classification d'images ... Les sujets sont aussi variés que les méthodes à explorer. Dans le cadre de cette formation, j'ai fait un stage pour Twice.ai, sur la segmentation sémantique d'image web. Ce travail, pour détecter des paragraphes textuels par vision par ordinateur a consisté en deux approches, la première basée sur du transfert d'apprentissage, la deuxième utilise la structure géométrique des paragraphes. Ce stage m'a permis de me rendre compte qu'il était temps pour moi de vendre mon travail : j'ai vraiment apprécié de répondre à besoin concret, avec des échéances courtes, dans un dynamisme d'entreprise. 

Les derniers projets de la formation ayant depuis été validés, je suis à la recherche d'un travail dans lequel investir mon temps et mes compétences. Mon profil est atypique, mais je sais que c'est une force : il prouve ma capacité d'adaptation, ma curiosité, et mon envie d'avancer. 
%
Je reste donc à votre entière disposition pour la suite, en espérant que mon profil puisse correspondre à vos attentes. \\
 
\end{document}