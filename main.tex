%%%%%%%%%%%%%%%%%
% This is an sample CV template created using altacv.cls
% (v1.1.4, 27 July 2018) written by LianTze Lim (liantze@gmail.com). Now compiles with pdfLaTeX, XeLaTeX and LuaLaTeX.
% 
%% It may be distributed and/or modified under the
%% conditions of the LaTeX Project Public License, either version 1.3
%% of this license or (at your option) any later version.
%% The latest version of this license is in
%%    http://www.latex-project.org/lppl.txt
%% and version 1.3 or later is part of all distributions of LaTeX
%% version 2003/12/01 or later.
%%%%%%%%%%%%%%%%

%% If you need to pass whatever options to xcolor
\PassOptionsToPackage{dvipsnames}{xcolor}

%% If you are using \orcid or academicons
%% icons, make sure you have the academicons 
%% option here, and compile with XeLaTeX
%% or LuaLaTeX.
% \documentclass[10pt,a4paper,academicons]{altacv}

%% Use the "normalphoto" option if you want a normal photo instead of cropped to a circle
% \documentclass[10pt,a4paper,normalphoto]{altacv}

\documentclass[10pt,a4paper]{altacv}
%% AltaCV uses the fontawesome and academicon fonts
%% and packages. 
%% See texdoc.net/pkg/fontawecome and http://texdoc.net/pkg/academicons for full list of symbols.
%% 
%% Compile with LuaLaTeX for best results. If you
%% want to use XeLaTeX, you may need to install
%% Academicons.ttf in your operating system's font 
%% folder.


% Change the page layout if you need to
\geometry{left=1cm,right=9cm,marginparwidth=6.8cm,marginparsep=1.2cm,top=1.25cm,bottom=1.25cm,footskip=2\baselineskip}

% Change the font if you want to.

% If using pdflatex:
\usepackage[T1]{fontenc}
\usepackage[utf8]{inputenc}
\usepackage[default]{lato}

% If using xelatex or lualatex:
% \setmainfont{Lato}

% Change the colours if you want to
\definecolor{HeadColor}{HTML}{70bdf1}%titres parties
\definecolor{AccentColor}{HTML}{4679bb}%sous-titre
\definecolor{EmphasisColor}{HTML}{012775}%prénom et info
\definecolor{LightGrey}{HTML}{666666}
\colorlet{heading}{HeadColor}
\colorlet{accent}{AccentColor}
\colorlet{emphasis}{EmphasisColor}
\colorlet{body}{LightGrey}

% Change the bullets for itemize and rating marker
% for \cvskill if you want to
\renewcommand{\itemmarker}{{\small\textbullet}}
\renewcommand{\ratingmarker}{\faCircle}
%% sample.bib contains your publications
\addbibresource{sample.bib}

\usepackage[colorlinks]{hyperref}
\usepackage{eurosym}

\begin{document}

\name{Claire GAYRAL}

\photo{2.8cm}{_MG_0628_01.jpg}
\tagline{PhD Student }
%\photo{2.8cm}{Globe_High}
\personalinfo{
    \email{claire.gayral69@gmail.com}
    \phone{+33 6 44 16 74 45}
  \\
    \mailaddress{33 rue de la filature, 69100 Villeurbanne}
    \car{Permis B}
    }


%% Make the header extend all the way to the right, if you want. 
\begin{fullwidth}
\makecvheader
\end{fullwidth}


\cvsection[page1sidebar]{Formation}

\cvevent{Master 2 : Mathématiques appliquées à la biologie et la médecine}{Université Lyon 1}{2018-2019}{Lyon, France}
\begin{itemize}
\item %TODO : multicols ?
    Analyse appliquée, Modélisation stochastique et statistique, Initiation au calcul scientifique intensif, Modélisation mathématique en épidémiologie, Dynamique de populations, Statistique pour la grande dimension en génomique, Anglais scientifique.
\end{itemize}

\cvevent{Master 1 : Mathématiques Appliquées et Statistiques}{Université Lyon 1}{2017-2018}{Lyon, France}
\begin{itemize}
\item %TODO : multicols ?
    Analyse (EDP, numérique, optimisation), Probabilités, Statistiques paramétriques, Logiciels Scientifiques, Séries Chronologiques, Traitement du signal, Recherche Opérationnelle, Projets divers.
\end{itemize}

\cvevent{Licence 3 de Mathématiques Générales}{Université de Montpellier}{2016-2017}{Montpellier, France}
% \begin{itemize}
% \item  %TODO : multicols ?
%     Théorie des Groupes, Topologie, Théorie de la Mesure, Analyse Numérique, Anglais, Calcul Différentiel, Probabilité et statistiques, Modélisation, Analyse Complexe, Optimisation.
% \end{itemize}

\cvevent{Classe Préparatoire aux Grandes Ecoles {\normalsize (BCPST1, PCSI, PC$^*$)}}{Lycée Louis Barthou }{2013-2016}{Pau, France}

% \begin{itemize}
% \item BCPST 1 : Spécialité Biologie Physique Chimie et Science de la Terre
% \item PCSI/PC$^*$ : Spécialité Physique Chimie, Classe étoilée.
% \end{itemize}



\cvsection{Expériences professionnelles}

\cvevent{Thèse statistiques appliquées à des données génomiques}{F. Picard LBBE (Lyon1) - J. Chiquet MIA (AgroParisTech)}{depuis Octobre 2019}{Lyon - Paris, France}
 Méthode d'intégration pour des données issues de séquençage en cellule unique
\begin{itemize}
\item Noyaux, Distances de Wasserstein, Méthodes algébriques (CCA)
\item Enseignement : 115h de TD de maths, niveau L1
\item Projet annexe d'intégration de données omiques (non cellule unique)
\end{itemize}

\cvevent{Stage de recherche}{F. Picard LBBE (Lyon1) - J. Chiquet MIA (AgroParisTech)}{Avril 2019- Août 2020}{Lyon - Paris, France}
E\'tude d'un jeu de données de séquençage en cellule unique (RNA-Seq et ATAC-Seq). 
\begin{itemize}
\item Analyse descriptive et pré-traitement (R) 
\item Essai d'intégration par méthode algébrique (CCA) 
\item Bibliographie sur la génomique et les méthodes en cellule unique
\end{itemize}

% \divider
\cvevent{Stage DataScientist}{S. Lousteau et C. Saumard, LumenAI}{Juin - Aout 2018}{Pau, France}

\begin{itemize}
\item Traitement de 3 projets clients (python), 
\item Maintenance prédictive, détection et caractérisations d'évènements suspects
\end{itemize}


\end{document}
    valeurs : humains - justice - respect.

