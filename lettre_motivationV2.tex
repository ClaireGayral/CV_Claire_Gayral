\documentclass[a4paper,12pt]{article}

\usepackage[T1]{fontenc}
\usepackage[utf8]{inputenc}
\usepackage[intlimits]{amsmath}
\usepackage{amsfonts,amssymb,amscd,amstext}


%%%% tikz library : %%%%
\usepackage{tikz}
\usetikzlibrary{tikzmark,calc,arrows,shapes,decorations.pathreplacing}
\usetikzlibrary{patterns,plotmarks}
\usetikzlibrary{decorations.pathreplacing,decorations.markings, arrows}
\newcommand{\midarrow}{\tikz \draw[-triangle 90] (0,0) -- +(.1,0);}
\usepackage{pgfplots}
\usepackage{caption,subcaption}
\tikzset{every picture/.style={remember picture}}


\usepackage{multicol}
\usepackage{enumerate}
\usepackage{array}
\usepackage{theorem}
\newtheorem{remark}{Remark}
\newtheorem{theorem}{Theorem}[subsection]

\newcommand{\aaa}{\alpha}
\newcommand{\bb}{\beta}


%ajouts ulysse
\newcommand{\Int}{\int\limits}
\newcommand{\om}{$\!\!^\sharp\,\,$}
\newcommand{\di}{$\!\!^\star\,\,$}
%\pagestyle{empty}
\newcommand{\ligneH}{\noindent\makebox[\linewidth]{\rule{\textwidth}{.5pt}}} %\paperwidth

%%%% mise en page : 
\pagestyle{plain} \setlength{\textwidth}{19cm}
\setlength{\textheight}{27cm} \setlength{\topmargin}{-3cm}
\setlength{\oddsidemargin}{-1,2cm}
\setlength{\evensidemargin}{+1,2cm}

\title{Motivation Letter}
\author{Claire Gayral}
\date{}

\begin{document}

\maketitle

Dear Mr. Seban, dear Mr. Redko,

Your multidisciplinary PhD thesis proposal, on the development of machine learning methods to understand bone marrow aging related to cosmic ray exposure, seems to fit perfectly with my professional project. I would therefore like to give you the reasons that push me to apply for this offer, which is one of the best opportunity for me, because it refers to different themes that I like. \\

The first point I need to motivate is why I want to carry out a PhD. The story begins during an optional internship, at the LumenAI company, in the first year of my Master's degree. I was supervised by Sébastien Lousteau and Camille Saumard. This internship made me realize that data mining would quickly become boring, because I was not experienced enough and the reality of the company  is not conducive for self-training. 

So I had to learn how to self-train. Especially since I was curious to understand the mathematics I used in the machine learning models.
And there, my thesis project was born through its aspect of research training. 
This 3-year project is also a challenge in one hand because of the stakes of publication, and in the other hand because of the modest contribution to science that it will bring. I am not yet completely fixed in my professional project following the PhD : looking for a teaching-research position, or going to work in a Research and Development department of a company. I will have time to mature this during my thesis (and see the opportunities at the end of it!). 

For my second year of Master, I thus opted for the research track, instead of the professionalization track. I graduated as the 7$^{\text{th}}$ of my promotion. 
My 4-month research internship was co-supervised by J.Chiquet and F.Picard. I read a lot about the biological notions of single cell genomics, and I cleaned and analyzed a data set, in order to understand regulatory mechanisms that would explain lymphocyte differentiation. Unfortunately, these data had too few signal to use them in an integration framework (3 measures per group to be identified for genomic expression, and 27 measures for 23 groups in regulation). This internship led, as agreed, to a PhD thesis on "Integration of single cell data". 
By integration, I mean finding manifold in which the data would have a small distance. For example, the Canonical Correlation Analysis (CCA) aims at finding a projection plane where the "correlation" distance is small. I got started with this classical method of integration, and its kernel and probabilistic extensions. Then to tackle the very particular distributions of these data (Zero-inflated, negative binomial, dependancy between measures ...), I looked at different metrics. Among other things, and I think you will understand why I insist with this, I spent a few months reading on Wasserstein metrics in their semi-discrete form (the objective being to replace the loss of maximum likelihood in CCA by a loss of minimum Wasserstein). In parallel with these bibliographical works, I have given lectures, on a little more than 120 hours of TD equivalent since January. 
So I have just finished this first thesis year, which has been more hard than challenging, because we have not been able to find a relevant communication with one of my directors. That is why it is better to reorient my PhD project toward a supervision that suits me better. 
Nevertheless, neither my skills nor the reasons that pushed me to undertake a PhD have been questioned (I let you contact my co-director with whom communication is clear, J. Chiquet, or my above-mentioned internship supervisor, S. Lousteau). 
The knowledge and work habits acquired during this year are assets that make me very competitive for another thesis project, and make this experience not a failure, but a springboard to this new project. 
%
Otherwise, my training would have allowed me to apply: the basics of signal processing acquired in M1, as well as the additionnal \textit{Machine Learning} course (from the master \textit{Data science} of Lyon 1, followed during my M2), give me the necessary baggage to launch me in this project. 
Otherwise, I could have applied one year ago, as I learned the required skills during my master degree, especially in the \textit{Signal processing} in M1 and the additionnal \textit{Machine Learning} course (from the master \textit{Data science} of Lyon 1, followed during my M2)
Moreover, my experience in the LumenAI company is a proof of my programming level, and my ability to handle Machine Learning tools in python. In addition to these skills, I learned this year how to use a remote server, within the framework of a project of integration of omics data (which were only.
In addition to these skills, I learned a lot this year. For example, I self-learned how to use a remote server, to carry out a side project (an omics data integration, on not single cell data), and run my R scripts. 


Why did I chose your offer in particular? The fact that it is a multidisciplinary work pleases me: the opportunity to discover biological curiosities makes sens to me. It is why I started my studies with a year of biology, and it is the same reason that made me choose the specialty "application to biology and medicine" of the Master \textit{Maths en Action}. This application of mathematics to medical problems seems to me constructive for science (much more than working on improving algorithms for targeted advertising!). 
From a technical point of view, I can assure you that I will be competent to meet the expectations for this project. In the description of the offer, you distinguished two axes for this thesis project. The second axis, on transfer methods, echoes what I read about the integration of omics data, which was often with the aim of using an a priori (gene expression), to conclude on another aspect (a regulation linked to the accessibility of DNA for example). The first axis, with which I am less familiar, would involve super-resolution methods. This reminds me of image processing courses that I had in the first year of my Master's degree, and I can't wait to discover this particular field! %, and the fact that I didn't pursue this interest was purely due to lack of opportunity. 
%
Moreover, the fact that the data is acquired and available from the very beginning of the work is, in my opinion, a great quality: it will allows us to target the biological problem(s) and to be able to test the adapted methods on concrete data, with all their particularities. Finally, I come from the region, and we would like to settle here again. It's a positive aspect, not decisive, but it's worth mentioning. 
\\
%

This is how I present myself to carry out this project with you. I am certain that it will be easy to build a relationship of trust at work, as long as the communication is clear and the scientific directions taken in a spirit of collaboration. I am therefore at your entire disposal to build this thesis, hoping that my application may correspond to your expectations. 
%



Translated with www.DeepL.com/Translator (free version)


\hspace{3cm}

\textbf{Contenu mail} : Monsieur Seban, Monsieur Redko, 




\end{document}
