
\PassOptionsToPackage{dvipsnames}{xcolor}

\documentclass[10pt,a4paper]{altacv}
%% AltaCV uses the fontawesome and academicon fonts
%% and packages. 
%% See texdoc.net/pkg/fontawecome and http://texdoc.net/pkg/academicons for full list of symbols.
%% 
%% Compile with LuaLaTeX for best results. If you
%% want to use XeLaTeX, you may need to install
%% Academicons.ttf in your operating system's font 
%% folder.


% Change the page layout if you need to
\geometry{left=1cm,right=9cm,marginparwidth=6.8cm,marginparsep=1.2cm,top=1.25cm,bottom=1.25cm,footskip=2\baselineskip}

% Change the font if you want to.

% If using pdflatex:
\usepackage[T1]{fontenc}
\usepackage[utf8]{inputenc}
\usepackage[default]{lato}

% If using xelatex or lualatex:
% \setmainfont{Lato}

% Change the colours if you want to
\definecolor{HeadColor}{HTML}{70bdf1}%titres parties
\definecolor{AccentColor}{HTML}{4679bb}%sous-titre
\definecolor{EmphasisColor}{HTML}{012775}%prénom et info
\definecolor{LightGrey}{HTML}{666666}
\colorlet{heading}{HeadColor}
\colorlet{accent}{AccentColor}
\colorlet{emphasis}{EmphasisColor}
\colorlet{body}{LightGrey}

% Change the bullets for itemize and rating marker
% for \cvskill if you want to
\renewcommand{\itemmarker}{{\small\textbullet}}
\renewcommand{\ratingmarker}{\faCircle}
%% sample.bib contains your publications
\addbibresource{sample.bib}

\usepackage[colorlinks]{hyperref}
\usepackage{eurosym}

\begin{document}

\name{Claire GAYRAL}

\photo{2.8cm}{_MG_0628_02.jpg}
\tagline{PhD Student }
%\photo{2.8cm}{Globe_High}
\personalinfo{
    \email{claire.gayral69@gmail.com}
    \phone{+33 6 44 16 74 45}
  \\
    \mailaddress{33 rue de la filature, 69100 Villeurbanne}
    \car{Permis B}
    }


%% Make the header extend all the way to the right, if you want. 
\begin{fullwidth}
\makecvheader
\end{fullwidth}


\cvsection[englishSidebar]{School Education}

\cvevent{Master 2 : Mathematics applied to biology and medicine}{Université Lyon 1}{2018-2019}{Lyon, France}
\begin{itemize}
\item 
    Applied Analysis, Stochastic and Statistical Modeling, Introduction to Intensive Scientific Computing, Mathematical Modeling in Epidemiology, Population Dynamics, Large-Scale Statistics in Genomics, Scientific English.
\end{itemize}

\cvevent{Master 1 : Applied Mathematics and Statistics}{Université Lyon 1}{2017-2018}{Lyon, France}
\begin{itemize}
\item 
    Analysis (EDP, numerical, optimization), Probability, Parametric Statistics, Scientific Software, Time Series, Signal Processing, Operational Research, Various Projects.
\end{itemize}

\cvevent{Licence 3 in General Mathematics}{Université de Montpellier}{2016-2017}{Montpellier, France}
% \begin{itemize}
% \item  %TODO : multicols ?
%     Théorie des Groupes, Topologie, Théorie de la Mesure, Analyse Numérique, Anglais, Calcul Différentiel, Probabilité et statistiques, Modélisation, Analyse Complexe, Optimisation.
% \end{itemize}

\cvevent{Classe Préparatoire aux Grandes Ecoles {\normalsize (BCPST1, PCSI, PC$^*$)}}{Lycée Louis Barthou }{2013-2016}{Pau, France}

% \begin{itemize}
% \item BCPST 1 : Spécialité Biologie Physique Chimie et Science de la Terre
% \item PCSI/PC$^*$ : Spécialité Physique Chimie, Classe étoilée.
% \end{itemize}



\cvsection{Work experiences in maths}

\cvevent{Statistical thesis applied to genomic data
}{F. Picard LBBE (Lyon1) - J. Chiquet MIA (AgroParisTech)}{since Octobre 2019}{Lyon - Paris, France}
 Integration method for single-cell data
\begin{itemize}
\item Kernel Methods, Wasserstein Distances, Algebraic Methods (CCA)
\item Teaching : $\sim$ 110h of math tutorials, level L1
\item Side project on omics data integration (non single-cell)
\end{itemize}

\cvevent{Research internship}{F. Picard LBBE (Lyon1) - J. Chiquet MIA (AgroParisTech)}{Avril 2019- Août 2020}{Lyon - Paris, France}
Study of a single cell sequencing data set (RNA-Seq and ATAC-Seq). 
\begin{itemize}
\item Descriptive analysis and pre-treatment (R) 
\item Attempt of Integration with Algebraic Method (CCA) 
\item Bibliography on Genomics and Single Cell Methods
\end{itemize}

% \divider
\cvevent{DataScientist Internship}{S. Lousteau et C. Saumard, LumenAI}{Juin - Aout 2018}{Pau, France}

\begin{itemize}
\item Processing of 3 client projects (python)
\item Predictive maintenance, detection and characterization of suspicious events
\end{itemize}


\end{document}
%    valeurs : humains - justice - respect.

