\documentclass[a4paper,11pt]{article}
\renewcommand{\baselinestretch}{0.94} 
\usepackage[T1]{fontenc}
\usepackage[utf8]{inputenc}
\usepackage[intlimits]{amsmath}
\usepackage{amsfonts,amssymb,amscd,amstext}


%%%% tikz library : %%%%
\usepackage{tikz}
\usetikzlibrary{tikzmark,calc,arrows,shapes,decorations.pathreplacing}
\usetikzlibrary{patterns,plotmarks}
\usetikzlibrary{decorations.pathreplacing,decorations.markings, arrows}
\newcommand{\midarrow}{\tikz \draw[-triangle 90] (0,0) -- +(.1,0);}
\usepackage{pgfplots}
\usepackage{caption,subcaption}
\tikzset{every picture/.style={remember picture}}


\usepackage{multicol}
\usepackage{enumerate}
\usepackage{array}
\usepackage{theorem}
\newtheorem{remark}{Remark}
\newtheorem{theorem}{Theorem}[subsection]

\newcommand{\aaa}{\alpha}
\newcommand{\bb}{\beta}


%ajouts ulysse
\newcommand{\Int}{\int\limits}
\newcommand{\om}{$\!\!^\sharp\,\,$}
\newcommand{\di}{$\!\!^\star\,\,$}
%\pagestyle{empty}
\newcommand{\ligneH}{\noindent\makebox[\linewidth]{\rule{\textwidth}{.5pt}}} %\paperwidth

%%%% mise en page : 
\pagestyle{plain} \setlength{\textwidth}{17,6cm}
\setlength{\textheight}{25cm} \setlength{\topmargin}{-2cm}
\setlength{\oddsidemargin}{-0.9cm}
\setlength{\evensidemargin}{+0.9cm}

\title{Lettre de motivation}
\author{Claire Gayral}
\date{}

\begin{document}

{\centering \Large \bf Paragraphe "le doctorant" \\ \vspace{0.6cm} }


Claire Gayral a une formation initiale en mathématiques appliquées. Elle est la $7^{eme}$ sur $29$ de la promotion 2019 du Master "Mathématiques en Action, appliquées à la biologie et la médecine" à l'Université Lyon 1, avec une moyenne de $14,33/20$. Ce Master, mêlant des étudiants de l'université avec des étudiants de l'école Centrale de Lyon et de l'École Nationale Supérieure de Lyon, forme un socle théorique solide pour une activité de recherche en mathématiques appliquées. Elle a ensuite travaillé pendant un an dans un contexte académique, entre le Laboratoire de Biométrie et Biologie Évolutive et le laboratoire de Mathématiques et Informatiques Appliquées. Son travail, ancré dans des statistiques en grande dimension pour la génomique, consistait à développer des méthodes d'intégration de données omiques issues de séquençage en cellule unique. Après une année de recherches bibliographiques, le sujet s'est avéré trop vague, elle a préféré réorienter son parcours vers des applications plus concrètes, et mieux définies. 
Elle a alors complété ses compétences de programmation python en suivant une formation "Ingénieur Machine Learning", validée par projets sur différentes problématiques : prédiction d’un score pour les aliments, prédiction de la consommation électrique de bâtiments, ségmentation de clients, catégorisation de questions, classification d’images, ... Dans le cadre de cette formation, elle a  travaillé sur l'outil UXvizer pendant 2 mois et demi, sur la segmentation sémantique d’image web. Ce travail, pour détecter des paragraphes textuels par vision par ordinateur a consisté en l'exploration de deux approches : la première basée sur du transfert d’apprentissage, la deuxième utilise la structure géométrique des paragraphes. Cette période de stage a permis d'organiser une rencontre entre les 3 parties du projet de thèse. 
Ce stage a été précédé par une première collaboration entre Claire Gayral et Camille Saumard, datant de 2018, dans l'entreprise LumenAI.  

Motivée par l'aspect de formation à la publication et d'expertise dans les domaines du ML, Claire Gayral semble être une candidate adaptée pour ce projet. Ses connaissances théoriques lui permettront de prendre en main le sujet, et sa montée en compétence en programmation sera utile pour tester les approches développées.\\ % et son récent travail de programmation permettra d'implémenter sans perdre de temps.

Durant la thèse, son travail sera réparti par période de deux ou trois mois au laboratoire, puis dans l'entreprise. La vie d'équipe dans l'autre structure (séminaire, réunion d'équipes) sera maintenue malgré l'alternance entre les deux structures. Des réunions de travail fixes incluant les trois parties seront prévues au moins une fois par mois, pour ajuster les directions du projet. %Le calendrier pourra être modifié selon l'avancement du projet.
La production de solutions adaptées au besoin de l'entreprise fera l'objet de publication sur des données publiques similaires. 




\textit{l’état de l’art, les objectifs principaux, la méthodologie de recherche, les moyens et matériels mis à disposition par chaque partenaire, des précisions sur l’organisation du travail du doctorant, un planning même préliminaire, ou encore d’éventuels travaux antérieurs, les raisons qui ont conduit à la sélection du candidat…. }

% motivation : 
% - besoin d'apprendre à publier
% - projet éthique 
% - application concrète
% - problématiques de propriété intellectuelle anticipée 


% contexte rassurant parce que : 
% - collaboration positive avec Camille 
% - communication très facile avec Romain, 
% - anticipation des problématiques telles que la propriété intellectuelle, 


\end{document}