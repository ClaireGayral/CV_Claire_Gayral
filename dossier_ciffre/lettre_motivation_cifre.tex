\documentclass[a4paper,11pt]{article}
\renewcommand{\baselinestretch}{0.94} 
\usepackage[T1]{fontenc}
\usepackage[utf8]{inputenc}
\usepackage[intlimits]{amsmath}
\usepackage{amsfonts,amssymb,amscd,amstext}


%%%% tikz library : %%%%
\usepackage{tikz}
\usetikzlibrary{tikzmark,calc,arrows,shapes,decorations.pathreplacing}
\usetikzlibrary{patterns,plotmarks}
\usetikzlibrary{decorations.pathreplacing,decorations.markings, arrows}
\newcommand{\midarrow}{\tikz \draw[-triangle 90] (0,0) -- +(.1,0);}
\usepackage{pgfplots}
\usepackage{caption,subcaption}
\tikzset{every picture/.style={remember picture}}


\usepackage{multicol}
\usepackage{enumerate}
\usepackage{array}
\usepackage{theorem}
\newtheorem{remark}{Remark}
\newtheorem{theorem}{Theorem}[subsection]

\newcommand{\aaa}{\alpha}
\newcommand{\bb}{\beta}


%ajouts ulysse
\newcommand{\Int}{\int\limits}
\newcommand{\om}{$\!\!^\sharp\,\,$}
\newcommand{\di}{$\!\!^\star\,\,$}
%\pagestyle{empty}
\newcommand{\ligneH}{\noindent\makebox[\linewidth]{\rule{\textwidth}{.5pt}}} %\paperwidth

%%%% mise en page : 
\pagestyle{plain} \setlength{\textwidth}{17,6cm}
\setlength{\textheight}{25cm} \setlength{\topmargin}{-2cm}
\setlength{\oddsidemargin}{-0.9cm}
\setlength{\evensidemargin}{+0.9cm}

\title{Lettre de motivation}
\author{Claire Gayral}
\date{}

\begin{document}

{\centering \Large \bf Lettre de motivation \\ \vspace{0.6cm} }

Le projet de thèse CIFRE sur l'Agrégation et le clustering de séries temporelles catégorielles m'intéresse, tant par les deux application sur l'analyse de parcours, que par les modélisations envisagées. Les objectifs de ces deux analyses s'ancrent dans une volonté éthique de l'utilisation du Machine Learning. Le sujet propose la juste dose entre mathématiques et informatique, avec une réelle motivation à la publication. Les deux approches proposées se réfèrent à des domaines différents, innovants et dont les applications sont très variées.%, la première s'appuyant sur l'alignement des séries temporelles, et l'autre sur les modélisations Markoviennes. 
Les données d'entraînement sont déjà à disposition, et les problématiques de publication et de propriété intellectuelles ont été anticipées par l'utilisation de deux applications aux caractéristiques communes. \\

Le projet de faire un doctorat est né par son aspect de formation à la recherche, lors de mon stage facultatif de première année de Master, encadré par Sébastien Lousteau et Camille Saumard. Ce stage  de Machine Learning se résume en le traitement des problématiques de détections de fraudes téléphoniques, et de maintenance prédictive, pour l'entreprise LumenAI. Il faisait appel à des outils variés, tant mathématiques (graphes, transport optimal, clustering, ...) que de programmation python (pandas, NetworkX, ...). 
Ce stage m'a permis de réaliser que la fouille de données avec les compétences que j'avais allait très rapidement m'ennuyer, car la réalité de l'entreprise n'en fait pas un milieu propice à l'auto-formation, dans l'état de mes compétences.
Pour la deuxième année de Master, la filière professionnalisante a été délaissée au profit de la filière recherche - année validée en étant 7$^{\text{ième}}$ de promotion.
Le stage de fin d'étude, de 4 mois, a été co-encadré par J. Chiquet et F. Picard. Il portait sur l'analyse de mesures d'expression et régulation génomique de lymphocytes, en cellule unique. Bien que les données n'avaient le signal suffisant pour publier une analyse, ce travail technique et bibliographique a mené à un projet de thèse sur "l'intégration de données de séquençage en cellule unique". La première année de thèse (du 01/10/2019 au 31/12/2020), a été consacrée à la bibliographie, et un projet d'intégration de données omiques. 
%
Intégration signifie ici trouver des plans de projection dans lesquels la distance est petite, comme l'Analyse Canonique des Corrélation (CCA) projette les données dans un espace où la distance "corrélation" est faible. Partant de cette méthode classique, j'ai étudié ses différentes extensions (à noyaux, probabiliste) puis j'ai cherché à remplacer la métrique par une mesure qui prendrait mieux en compte la distribution si particulière des données (différentes divergences, transport optimal). 
% 
En parallèle de ces travaux, j'ai dispensé 110 heures de TD en 2020, pour l'UFR de mathématiques de l'UCBL. 
Malheureusement, cette année de thèse ne s'est pas déroulée dans des conditions optimales, notamment à cause de la définition trop vague du sujet de thèse. 
%
Il a donc été décidé d'interrompre ce projet. %, bien que ni mes compétences, ni les raisons qui me poussaient à entreprendre un doctorat n'ont été remises en cause (je vous laisse le soin de contacter mon ancien directeur J. Chiquet, ou encore mon encadrant de stage de M1, S. Lousteau).
%
Les connaissances apprises et les habitudes de travail acquises durant cette année font de cette expérience un tremplin vers un projet qui me conviendrait mieux. En particulier, une application plus concrète, avec une dynamique d'entreprise semblerait mieux me correspondre. La formation complémentaire "Ingénieur Machine Learning" d'OpenClassRooms était alors la clé pour rapidement apprendre à développer des projets en python, avec les bonnes pratiques de programmation et les bibliothèques adaptées. 
Cette formation en ligne est basée sur des projets avec différentes problématiques : prédiction d'un score pour les aliments, prédiction de la consommation électrique de bâtiments, segmentation de clients, catégorisation de questions et prédiction de tags associés, classification d'images ... Les sujets sont aussi variés que les méthodes à explorer. Dans le cadre de cette formation, j'ai fait un stage pour Twice.ai, encadré par Camille Saumard, sur la segmentation sémantique d'image web. Ce travail, pour détecter des paragraphes textuels par vision par ordinateur m'a permis d'explorer deux approches pour répondre à cette problématique, la première basée sur du transfert d'apprentissage, la deuxième utilise la structure géométrique des paragraphes. 
%
Pour autant, la valorisation du travail scientifique par la publication, et l'acquisition d'une expertise dans un domaine précis restent des objectifs que je pense atteindre par une thèse. 
%
Je ne suis pas encore figée dans mon projet professionnel au long terme : poursuivre un parcours académique, ou travailler dans une entreprise, sur des projets de Recherches et Développement.
%Je pense avoir le temps de maturer cela au cours de la thèse. 
%
\\



L'opportunité qu'offre ce projet est particulièrement adaptée à mon projet professionnel. En effet, l'entreprise a bien défini ses besoins et attentes, fourni les données de parcours utilisateur à analyser. En parallèle, le laboratoire propose un cadre académique riche, avec un encadrement qui se veut adapté aux besoins du projet, et permettra la publication sur les données de trajectoires de bateaux. 
Une première rencontre avec les différentes parties du projet a permis de fixer le sujet, les axes scientifiques, et les besoins individuels. Si les deux expériences très positives de stage avec Camille Saumard assurent une communication d'encadrement déjà établie, le contact avec Romain Tavenard a été particulièrement intéressant, par les idées proposées et la délimitation des attentes respectives. 
Bien que mes travaux de recherches précédents diffèrent du sujet, ma formation en mathématiques et informatique me donne la flexibilité et les compétences nécessaires pour aborder cette thématique. %La première expérience académique m'a donné une méthodologie de travail que je saurai remettre à profit. 
Je reste donc à votre entière disposition pour construire cette thèse, en espérant que ma candidature puisse correspondre à vos attentes. \\

Meilleures salutations, \\

Claire Gayral


\end{document}