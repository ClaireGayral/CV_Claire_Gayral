\documentclass[a4paper,11pt]{article}

\usepackage[T1]{fontenc}
\usepackage[utf8]{inputenc}
\usepackage[intlimits]{amsmath}
\usepackage{amsfonts,amssymb,amscd,amstext}


%%%% tikz library : %%%%
\usepackage{tikz}
\usetikzlibrary{tikzmark,calc,arrows,shapes,decorations.pathreplacing}
\usetikzlibrary{patterns,plotmarks}
\usetikzlibrary{decorations.pathreplacing,decorations.markings, arrows}
\newcommand{\midarrow}{\tikz \draw[-triangle 90] (0,0) -- +(.1,0);}
\usepackage{pgfplots}
\usepackage{caption,subcaption}
\tikzset{every picture/.style={remember picture}}


\usepackage{multicol}
\usepackage{enumerate}
\usepackage{array}
\usepackage{theorem}
\newtheorem{remark}{Remark}
\newtheorem{theorem}{Theorem}[subsection]

\newcommand{\aaa}{\alpha}
\newcommand{\bb}{\beta}


%ajouts ulysse
\newcommand{\Int}{\int\limits}
\newcommand{\om}{$\!\!^\sharp\,\,$}
\newcommand{\di}{$\!\!^\star\,\,$}
%\pagestyle{empty}
\newcommand{\ligneH}{\noindent\makebox[\linewidth]{\rule{\textwidth}{.5pt}}} %\paperwidth

%%%% mise en page : 
\pagestyle{plain} \setlength{\textwidth}{17,6cm}
\setlength{\textheight}{25cm} \setlength{\topmargin}{-2cm}
\setlength{\oddsidemargin}{-0.9cm}
\setlength{\evensidemargin}{+0.9cm}

\title{Lettre de motivation}
\author{Claire Gayral}
\date{}

\begin{document}

{\centering \Large \bf Lettre de motivation \\ \vspace{0.6cm} }

Monsieur Haddad, Madame Seba,\\

Votre offre de stage de master 2, sur la détection d’anomalies dans un flux de graphes a attiré mon attention, car cette thématique de recherche, avec de multiples applications, me semble aussi intéressante que valorisable.

\textcolor{red}{Je suis plus intéressée par l'offre de thèse accompagnant le stage que le stage lui-même, et j'aimerais vous détailler les raisons qui m'amènent à vous soumettre ma candidature.}

Bien que n'étant pas particulièrement à la recherche d'un stage de Master 2, mais plutôt d'une thèse, 
je me permets de vous exposer les raisons qui me poussent à postuler.%, qui est une opportunité de choix pour moi, parce qu'elle fait écho à différentes thématiques qui me plaisent. \\

Premièrement, mon projet de faire un doctorat est né par son aspect de formation à la recherche, lors de mon stage facultatif de fin de première année de Master en Machine Learning.%, pendant lequel j'ai été encadrée par Sébastien Lousteau et Camille Brunet-Saumard.
J'ai eu à traiter des problématiques de détections de fraude téléphonique, et de maintenance prédictive pour l'entreprise LumenAI. \textcolor{red}{citer Lousteau ?} J'ai découvert de nombreux outils, tant mathématiques (graphes, transport optimal, clustering, ...) que de programmation (pandas, NetworkX, ...). %Au delà de la montée en compétences, c
Ce stage m'a permis de réaliser que la fouille de données avec les compétences que j'avais allait très rapidement m'ennuyer, car la réalité de l'entreprise n'en fait pas un milieu propice à l'auto-formation.
%%
%Il m'était donc nécessaire d'apprendre à apprendre
% par moi-même. %D'autant plus que ma curiosité %pour les mathématiques se cachant derrière 
%pour les modèles d'apprentissage statistique restaient à assouvir.
%%
%Ce projet sur 3 ans est aussi un défi de par les enjeux de publication%, et de façon plus globale par le modeste apport à la science qu'il amènera. 
%
Pour ma deuxième année de Master, j'ai donc choisi quitter la filière professionnalisante, au profit de la filière recherche - année validée en étant 7$^{\text{ième}}$ de promotion. 
J'ai enchaîné sur 4 mois de stage de recherche co-encadré par J. Chiquet et F. Picard, durant lesquels je me suis familiarisée avec les notions biologiques de génomique en cellule unique, et j'ai nettoyé et analysé un jeu de données, dans l'objectif de révéler des mécanismes de régulation responsables de la différentiation de lymphocyte. Ces données avaient trop peu de signal pour les utiliser dans un cadre d'intégration, à valoriser dans une publication. \textcolor{red}{J'ai entamé une première année de thèse, portant sur le sujet...} S'en est suivi une thèse, portant sur le sujet "intégration de données de séquençage en cellule unique", dont la première année a été consacrée à la bibliographie, et un projet d'intégration de données omiques. %Par intégration, j'entends trouver des plans de projection dans lesquels les données aurait une distance faible, comme l'Analyse Canonique des Corrélation (CCA) est une méthode algébrique qui cherche un plan de projection où la distance "corrélation" est faible. J'ai commencé m'intéresser à cette méthode classique d'intégration, et ses extensions à noyaux, et probabilistes, puis j'ai regardé quelles métriques pourraient être utilisées pour intégrer ces données avec des distributions si particulières (différentes divergences, et distance de Wasserstein).
En parallèle de ces travaux, j'ai dispensé %donné des enseignements, 
110 heures de TD depuis janvier 2020, pour l'UFR de mathématiques de l'UCBL. 

\textcolor{red}{Malheureusement, cette première année de thèse ne s'est pas déroulée dans des conditions optimales, notamment à cause de problèmes de communication avec un de mes directeurs....}

Je suis au terme de cette première année de thèse, qui a été une année très désagréable, car nous n'avons pas réussi à trouver une communication constructive avec un de mes directeurs, ce qui rendait mon travail de plus en plus compliqué. Il a donc été décidé d'interrompre ce projet, bien que ni mes compétences, ni les raisons qui me poussaient à entreprendre un doctorat n'ont été remises en cause (je vous laisse le soin de contacter mon directeur actuel J. Chiquet, ou encore mon encadrant de stage de M1, S. Lousteau).
Les connaissances apprises et les habitudes de travail acquises durant cette année %sont des atouts %qui me rendent très compétitive pour un autre projet de thèse, et 
font de cette expérience non un échec, mais bien un tremplin vers un projet qui me conviendrait mieux. 
%
Je ne suis pas encore figée dans mon projet professionnel au long terme : poursuivre un parcours académique, ou partir en entreprise. Je pense avoir le temps de maturer cela au cours de la thèse (et voir les opportunités à la fin de celle-ci !).  \textcolor{red}{pas utile je pense}
%
%Ma formation m'aurait par ailleurs permis de postuler à votre offre sans cette année supplémentaire : j'ai suivi le cours de M. Haddad sur la \textit{Recherche Opérationnelle} en M1, ainsi que le cours de \textit{Machine Learning}, et quelques cours de \textit{Probabilistic Graphical Models} (du master \textit{Data science} de Lyon 1, assurés par M. Aussem, que suivis en complément de mon M2)% m'ont donné le bagage nécessaire pour me lancer dans ce projet. 
%J'ai également eu l'occasion de lire quelques références sur l'inférence de réseaux dynamiques lors d'un petit projet d'anglais scientifiques en Master 2 (rapport ci-joint). Enfin, mon expérience dans l'entreprise LumenAI fait preuve de mon niveau de programmation, et de ma capacité à prendre facilement en main les outils de Machine Learning en python. En plus de ces compétences, dans le cadre de cette année de thèse, j'ai eu la chance de pouvoir participer à l'école ECAS sur l'analyse, l'inférence et les modélisation de graphes. J'ai également appris en autodidacte à utiliser un serveur distant, dans le cadre d'un projet d'intégration de données omiques, pour lancer mes scripts R. \\
%
%

\textcolor{red}{Bien que mes travaux de recherches précédents diffèrent du sujet que vous proposez, ma formation en mathématiques et informatique me donne la flexibilité nécessaire pour aborder cette thématique différente; et mon expérience de thèse m'a donné une méthodologie de travail que je saurais remettre à profit.}


Voilà donc mon profil et les raisons qui m'amènent à chercher une thèse aujourd'hui. \textcolor{red}{phrase précédente pas utile} Je l'ai bien noté, vous proposez une offre de stage de Master et non de thèse. Mon expérience récente me pousse à vouloir faire quelques mois de travail bibliographique avec vous avant de m'engager dans un projet de thèse : \textcolor{red}{je ne doute pas que l'on puisse construire ensemble une relation de confiance pour travailler, mais mon expérience récente m'incline à prendre le temps de bâtir une telle relation avant de m'engager dans un projet à long terme. Je suis sûre que vous comprendrez, étant donné les circonstances.}


Ce pourquoi une forme de "stage" me paraît tout à fait appropriée.  L'offre est claire, et propose une bibliographie assez riche comme point de départ. 
%
Pourquoi cette offre en particulier ? Premièrement, j'ai eu l'occasion de découvrir les graphes sous plusieurs aspects : combinatoire
%(cours de \textit{recherche opérationnelle})
, probabiliste
%(cours d'\textit{épidémiologie}, de \textit{Probabilistic Graphical Models})
, statistique %(projet d'anglais scientifique, et formation ECAS)
, computationnel 
%(stage de master 1) 

\textcolor{red}{à mon avis, pas besoin d'énumérer en quoi tu connais les graphes : c'est plus ou moins la base et tu as pas à te jusitifer là dessus}

\textcolor{red}{same pour ce qui suit : pas besoin de dire que t'es déjà forte sur le sujet, ton CV parle pour toi et/ou l'entretien convaincra}

. D'un autre côté, j'ai eu l'occasion de m'intéresser aux problématiques de détection d'anomalies, %et j'ai une idée de la quantité d'application possible, 
ce qui en rend l'outil d'autant plus intéressant. 
%D'un point de vue technique, je peux vous affirmer que je serai compétente pour répondre aux attentes pour ce projet. Pour la lecture d'article, cette année de thèse m'a appris à lire et comprendre efficacement une littérature d'un domaine qui m'était étranger, à aller chercher les références dont j'avais besoin pour avancer, à cibler les questions non traitées qui pourraient devenir des pistes de recherche. Le sujet de lecture, bien que loin d'être facile, semble largement accessible à ma compréhension. Et la programmation ne me pose pas de soucis, c'est même quelque chose que j'apprécie. J'ai donc hâte de pouvoir me lancer dans ce nouveau projet.

%Enfin, je suis originaire de la région, et nous souhaitons nous installer ici avec mon conjoint. C'est un point positif, non décisif, mais qui mérite d'être mentionné. 
%

C'est ainsi que je me présente pour mener à bien ce projet bibliographique avec vous \textcolor{red}{ils ont pas encore accepté : mettre plutôt : j'espère que ce projet bibliographique, avec l'idée d'en faire une thèse ensuite, est une option qui vous conviendrait. je suis certain blablaba}, dans l'idée de le transformer en projet de thèse après l'avoir mené à bout. Je suis certaine qu'il sera facile de construire une relation de confiance dans le travail, dès lors que la communication est claire, et les directions scientifiques prises dans un esprit de collaboration. Je reste donc à votre entière disposition pour construire ce travail bibliographique, et cette thèse, en espérant que ma candidature puisse correspondre à vos attentes. \\

Meilleures salutations, \\

Claire Gayral

\vspace{2cm}
\textbf{Contenu mail} : 


Monsieur Haddad, Madame Seba,

Je vous contacte suite à la lecture de l'offre de stage sur la détection d'anomalie dans un flux de graphes. Serait-il  envisageable de prévoir un rendez-vous téléphonique ? 

Je suis actuellement à la recherche plus d'une thèse que d'un stage, mais le sujet que vous proposez a l'air de vraiment me plaire. N'étant pas dans une formation diplômante, il ne m'est pas possible de faire un stage sous sa forme conventionnelle, mais j'ai la chance de pouvoir vous proposer quelques mois de travail bibliographiques. 

Je vous présente mon profil dans la lettre de motivation et le CV ci-joints. Si vous souhaitez de plus amples informations, n'hésitez pas à me les demander.  

Bien cordialement, 

Claire Gayral

\end{document}

\textbf{Contenu mail} : 


Monsieur Haddad, Madame Seba,

Je vous contacte suite à la lecture de l'offre de stage sur la détection d'anomalie dans un flux de graphes. Serait-il  envisageable de prévoir un rendez-vous téléphonique ? 

Je suis actuellement à la recherche plus d'une thèse que d'un stage, mais le sujet que vous proposez a l'air de vraiment me plaire. N'étant pas dans une formation diplômante, il ne m'est pas possible de faire un stage sous sa forme conventionnelle, mais j'ai la chance de pouvoir vous proposer quelques mois de travail bibliographiques. 

Si vous souhaitez de plus amples informations, n'hésitez pas à me les demander. 

Bien cordialement, 

Claire Gayral




Je vous contacte au sujet de l'offre de thèse intitulée "Machine learning for understanding the aging of a human bone due to space exposure", qui pourrait tout à fait correspondre à mon projet professionnel. Comme exigé dans l'offre, je vous joins donc ma lettre de motivation et mon CV, en anglais et français.

Je vous laisse les contacts de différentes personnes, qui ont eu l'occasion de m'encadrer, qui ont gentiment proposé d'être référents : 
- Julien Chiquet, tel : +33(0)1 44 08 18 39; mail :julien.chiquet@inrae.fr; site : http://julien.cremeriefamily.info/
- Sébastien Lousteau : tel : +33 7 83 27 38 83, mail : sloustau@lumenai.fr, site : https://twitter.com/seb_loustau
- Clément Marteau : 
- Camille Saumard : 

