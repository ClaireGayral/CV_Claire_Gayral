\documentclass[a4paper,12pt]{article}

\usepackage[T1]{fontenc}
\usepackage[utf8]{inputenc}
\usepackage[intlimits]{amsmath}
\usepackage{amsfonts,amssymb,amscd,amstext}


%%%% tikz library : %%%%
\usepackage{tikz}
\usetikzlibrary{tikzmark,calc,arrows,shapes,decorations.pathreplacing}
\usetikzlibrary{patterns,plotmarks}
\usetikzlibrary{decorations.pathreplacing,decorations.markings, arrows}
\newcommand{\midarrow}{\tikz \draw[-triangle 90] (0,0) -- +(.1,0);}
\usepackage{pgfplots}
\usepackage{caption,subcaption}
\tikzset{every picture/.style={remember picture}}


\usepackage{multicol}
\usepackage{enumerate}
\usepackage{array}
\usepackage{theorem}
\newtheorem{remark}{Remark}
\newtheorem{theorem}{Theorem}[subsection]

\newcommand{\aaa}{\alpha}
\newcommand{\bb}{\beta}


%ajouts ulysse
\newcommand{\Int}{\int\limits}
\newcommand{\om}{$\!\!^\sharp\,\,$}
\newcommand{\di}{$\!\!^\star\,\,$}
%\pagestyle{empty}
\newcommand{\ligneH}{\noindent\makebox[\linewidth]{\rule{\textwidth}{.5pt}}} %\paperwidth

%%%% mise en page : 
\pagestyle{plain} \setlength{\textwidth}{17,6cm}
\setlength{\textheight}{25cm} \setlength{\topmargin}{-2cm}
\setlength{\oddsidemargin}{-0.9cm}
\setlength{\evensidemargin}{+0.9cm}

\title{Lettre de motivation}
\author{Claire Gayral}
\date{}

\begin{document}

\maketitle

Monsieur Haddad, Madame Seba,\\

Votre offre de stage de master 2, sur la détection d’anomalies dans un flux de graphes a attiré mon attention, car cette thématique de recherche, avec de multiples applications, me semble aussi intéressante que valorisable. L'offre est bien documentée, et propose une bibliographie assez riche comme point de départ. Bien que n'étant pas particulièrement à la recherche d'un stage de Master 2, mais plutôt d'une thèse, je me permets de vous exposer les raisons qui me poussent à postuler à cette offre.%, qui est une opportunité de choix pour moi, parce qu'elle fait écho à différentes thématiques qui me plaisent. \\

La première chose qu'il me semble nécessaire à motiver est mon envie de faire une thèse. L'historique débute lors d'un stage facultatif en machine learning, fait dans l'entreprise LumenAI, en fin de première année de Master, pendant lequel j'ai été encadrée par Sébastien Lousteau et Camille Brunet-Saumard. Cette expérience durant laquelle j'ai eu à traiter, par exemple, des problématiques de détections de fraude téléphonique, m'a permis de découvrir de nombreux outils, tant mathématiques (ex : graphes, transport optimal) que de programmation (ex : pandas, NetworkX). Bien plus que la montée en compétences, ce stage m'a permis de réaliser que la fouille de données avec les compétences que j'avais allait très rapidement m'ennuyer, car la réalité de l'entreprise n'en fait pas un milieu propice à l'auto-formation. 

Il m'était donc nécessaire d'apprendre à apprendre par moi-même. D'autant plus que ma curiosité pour les mathématiques se cachant derrière les modèles d'apprentissage statistique restaient à assouvir.
C'est ainsi qu'est né ce projet de thèse, par son aspect de formation à la recherche. 
Ce projet sur 3 ans est aussi un défi de par les enjeux de publication, et de façon plus globale par le modeste apport à la science qu'il amènera. Je ne suis pas encore complètement figée dans mon projet professionnel après cette étape : chercher un poste d'enseignant-chercheur, ou partir dans un service Recherches et Développement d'une entreprise. Je pense avoir le temps de maturer cela au cours de la thèse (et voir les opportunités à la fin de celle-ci !). 

Pour ma deuxième année de Master, j'ai donc choisi quitter la filière plus professionnalisante, au profit de la filière recherche - année validée en étant 7$^{\text{ième}}$ de ma promotion. 
J'ai enchaîné sur 4 mois de stage de recherche co-encadré par Julien Chiquet et Franck Picard, durant lesquels je me suis familiarisée avec les notions biologiques de génomique en cellule unique, et j'ai nettoyé et analysé un jeu de données, dans l'objectif de comprendre des mécanismes de régulation qui expliqueraient la différentiation de lymphocyte. Malheureusement, ces données avaient trop peu de signal pour les utiliser dans un cadre d'intégration. Ce stage a débouché, comme il en avait été convenu, sur un contrat doctoral, portant sur le sujet "intégration de données de séquençage en cellule unique". Par intégration, j'entends trouver des plans de projection dans lesquels les données aurait une distance faible, comme l'Analyse Canonique des Corrélation (CCA) est une méthode algébrique qui cherche un plan de projection où la distance "corrélation" est faible. J'ai commencé m'intéresser à cette méthode classique d'intégration, et ses extensions à noyaux, et probabilistes, puis j'ai regardé quelles métriques pourraient être utilisées pour intégrer ces données avec des distributions si particulières (différentes divergences, et distance de Wasserstein).% J'ai notamment consacré quelques mois à me familiariser d'un point de vue théorique avec les distances de Wasserstein, dans leur forme semi-discrète (l'objectif étant de remplacer la perte de maximum de vraissemblance dans la CCA par une perte de minimum de Wasserstein). Au vu de vos récentes publications respectives, j'espère que le fait de ne pas être ingorante sur les problématiques de transport optimal sera appréciable. 

En parallèle de ces travaux bibliographiques, j'ai donné des enseignements, sur un peu plus de 110 heures équivalent TD depuis le mois de janvier 2020, pour l'UFR de mathématiques de l'UCBL. 
Je suis au terme de cette première année de thèse, qui a été une année très désagréable, car nous n'avons pas réussi à trouver une communication constructive avec un de mes directeurs, ce qui a rendu le travail de plus en plus compliqué. Il a donc été décidé d'interrompre ce projet, bien que ni mes compétences, ni les raisons qui me poussaient à entreprendre un doctorat n'ont été remises en cause (je vous laisse le soin de contacter mon co-directeur avec qui la communication est limpide, J. Chiquet, ou encore mon encadrant de stage de M1, S. Lousteau).
Les connaissances apprises et les habitudes de travail acquises durant cette année %sont des atouts %qui me rendent très compétitive pour un autre projet de thèse, et 
font de cette expérience non un échec, mais bien un tremplin vers un projet qui me conviendrait mieux. 
%
%Ma formation m'aurait par ailleurs permis de postuler à votre offre sans cette année supplémentaire : j'ai suivi le cours de M. Haddad sur la \textit{Recherche Opérationnelle} en M1, ainsi que le cours de \textit{Machine Learning}, et quelques cours de \textit{Probabilistic Graphical Models} (du master \textit{Data science} de Lyon 1, assurés par M. Aussem, que suivis en complément de mon M2)% m'ont donné le bagage nécessaire pour me lancer dans ce projet. 
%J'ai également eu l'occasion de lire quelques références sur l'inférence de réseaux dynamiques lors d'un petit projet d'anglais scientifiques en Master 2 (rapport ci-joint). Enfin, mon expérience dans l'entreprise LumenAI fait preuve de mon niveau de programmation, et de ma capacité à prendre facilement en main les outils de Machine Learning en python. En plus de ces compétences, dans le cadre de cette année de thèse, j'ai eu la chance de pouvoir participer à l'école ECAS sur l'analyse, l'inférence et les modélisation de graphes. J'ai également appris en autodidacte à utiliser un serveur distant, dans le cadre d'un projet d'intégration de données omiques, pour lancer mes scripts R. \\
%

Voilà donc mon profil et les raisons qui m'amènent à chercher une thèse aujourd'hui. Cependant, je l'ai bien noté, vous proposez une offre de stage de Master et non de thèse. Il est clair que mon expérience récente me pousse à vouloir faire quelques mois de travail bibliographique avec vous avant de m'engager dans un autre projet de thèse : c'est une façon de vous rassurer, et de me rassurer, sur le fait que l'on travaillera en confiance (ce qui est presque sûr, c'était ma première expérience d'un encadrement aussi compliqué). Ce pourquoi une forme de "stage" me paraît tout à fait appropriée.

Pourquoi cette offre en particulier ? Comme je l'évoquais, j'ai eu l'occasion de découvrir les graphes sous plusieurs aspects : combinatoire (cours de \textit{recherche opérationnelle}), probabiliste (cours d'\textit{épidémiologie}), statistique (projet d'anglais scientifique, et formation ECAS), computationnel (stage de master 1). Sans dire que je suis experte de cet objet, il m'est familier. D'un autre côté, j'ai eu l'occasion de m'intéresser aux problématiques de détection d'anomalies, et j'ai une idée de la quantité d'application possible, ce qui en rend l'outil d'autant plus intéressant. 
D'un point de vue technique, je peux vous affirmer que je serai compétente pour répondre aux attentes pour ce projet. Pour la lecture d'article, cette année de thèse m'a appris à lire et comprendre efficacement une littérature d'un domaine qui m'était étranger, à aller chercher les références dont j'avais besoin pour avancer, à cibler les questions non traitées qui pourraient devenir des pistes de recherche. Le sujet de lecture, bien que loin d'être facile, semble largement accessible à ma compréhension. Et la programmation ne me pose pas de soucis, c'est même quelque chose que j'apprécie. J'ai donc hâte de pouvoir me lancer dans ce nouveau projet.

Enfin, je suis originaire de la région, et nous souhaitons nous installer ici avec mon conjoint. C'est un point positif, non décisif, mais qui mérite d'être mentionné. 
\\
%

C'est ainsi que je me présente pour mener à bien ce projet bibliographique avec vous, dans l'idée de le transformer en projet de thèse après l'avoir mené à bout. Je suis certaine qu'il sera facile de construire une relation de confiance dans le travail, dès lors que la communication est claire, et les directions scientifiques prises dans un esprit de collaboration. Je reste donc à votre entière disposition pour construire ce travail bibliographique, et cette thèse, en espérant que ma candidature puisse correspondre à vos attentes. \\

Meilleures salutations, \\

Claire Gayral

\vspace{2cm}
\textbf{Contenu mail} : 


Monsieur Haddad, Madame Seba,

Je vous contacte suite à la lecture de l'offre de stage sur la détection d'anomalie dans un flux de graphes. Serait-il  envisageable de prévoir un rendez-vous téléphonique ? 

Je suis actuellement à la recherche plus d'une thèse que d'un stage, mais le sujet que vous proposez a l'air de vraiment me plaire. N'étant pas dans une formation diplômante, il ne m'est pas possible de faire un stage sous sa forme conventionnelle, mais j'ai la chance de pouvoir vous proposer quelques mois de travail bibliographiques. 

Je vous présente mon profil dans la lettre de motivation et le CV ci-joints. Si vous souhaitez de plus amples informations, n'hésitez pas à me les demander.  

Bien cordialement, 

Claire Gayral

\end{document}

\textbf{Contenu mail} : 


Monsieur Haddad, Madame Seba,

Je vous contacte suite à la lecture de l'offre de stage sur la détection d'anomalie dans un flux de graphes. Serait-il  envisageable de prévoir un rendez-vous téléphonique ? 

Je suis actuellement à la recherche plus d'une thèse que d'un stage, mais le sujet que vous proposez a l'air de vraiment me plaire. N'étant pas dans une formation diplômante, il ne m'est pas possible de faire un stage sous sa forme conventionnelle, mais j'ai la chance de pouvoir vous proposer quelques mois de travail bibliographiques. 

Si vous souhaitez de plus amples informations, n'hésitez pas à me les demander. 

Bien cordialement, 

Claire Gayral




Je vous contacte au sujet de l'offre de thèse intitulée "Machine learning for understanding the aging of a human bone due to space exposure", qui pourrait tout à fait correspondre à mon projet professionnel. Comme exigé dans l'offre, je vous joins donc ma lettre de motivation et mon CV, en anglais et français.

Je vous laisse les contacts de différentes personnes, qui ont eu l'occasion de m'encadrer, qui ont gentiment proposé d'être référents : 
- Julien Chiquet, tel : +33(0)1 44 08 18 39; mail :julien.chiquet@inrae.fr; site : http://julien.cremeriefamily.info/
- Sébastien Lousteau : tel : +33 7 83 27 38 83, mail : sloustau@lumenai.fr, site : https://twitter.com/seb_loustau
- Clément Marteau : 
- Camille Saumard : 

